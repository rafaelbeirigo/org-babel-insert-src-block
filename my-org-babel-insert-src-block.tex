% Created 2022-11-12 Sat 01:43
% Intended LaTeX compiler: pdflatex
\documentclass[11pt]{article}
\usepackage[dvipsnames]{xcolor}
\usepackage[utf8]{inputenc}
\usepackage[T1]{fontenc}
\usepackage{graphicx}
\usepackage{grffile}
\usepackage{longtable}
\usepackage{wrapfig}
\usepackage{rotating}
\usepackage[normalem]{ulem}
\usepackage{amsmath}
\usepackage{textcomp}
\usepackage{amssymb}
\usepackage{capt-of}
\usepackage[colorlinks=true,allcolors=NavyBlue]{hyperref}
\usepackage{verbatim}
\newlength\myverbindent
\setlength\myverbindent{1.1em} % change this to change indentation
\makeatletter
\def\verbatim@processline{%
\hspace{\myverbindent}\the\verbatim@line\par}
\makeatother
\author{Rafael Beirigo}
\date{\today}
\title{}
\hypersetup{
 pdfauthor={Rafael Beirigo},
 pdftitle={},
 pdfkeywords={},
 pdfsubject={},
 pdfcreator={Emacs 28.2 (Org mode 9.5.5)}, 
 pdflang={English}}
\begin{document}

\tableofcontents


\section{Insert Code Block---Ask for a Name}
\label{sec:org63cfe8d}

This project is a function for Org Babel that, before inserting a
source code block, asks the user for a name.  If the user does not
inform a name, nothing changes.  But if the user does inform a name,
the function will automatically
\begin{itemize}
\item decorate the block with a nicelly formatted string,
\item insert a \texttt{:noweb-ref} for the block, and
\item create an Org heading for the code block section.
\end{itemize}

The idea is to make it easier to do \href{https://en.wikipedia.org/wiki/Literate\_programming}{literate programming} on \href{https://en.wikipedia.org/wiki/GNU\_Emacs}{Emacs}.

All the text below \emph{is} literate programming (LP) itself: the ideas
and divagations are mixed with the source code, and the final product
is automatically extracted from this text via \emph{tangling}.

Thus, the whole entire source code for the program is listed below!
For me, this was the first not-so-short project that I used LP, and I
loved it!

\subsection{Introduction}
\label{sec:org449be63}

Org Babel has this nice little functionality of

\begin{quote}
Insert[ing] a block structure of the type \texttt{\#+begin\_foo/\#+end\_foo}.
\end{quote}

using the function \texttt{org-insert-structure-template}.  This is very
practical, because I have to type less.

What I \emph{wish} Org Babel had was a way to insert the \emph{name} of the code
blocks.  This would help a lot in my literate programming activities.

\begin{verbatim}
<<Function to insert an Org Babel code block>>
\end{verbatim}

\subsection{Function to insert an Org Babel code block}
\label{sec:orgc246152}

The code block may or may not have a name, and this will determine
several things about the tangling process.  The general structure of
the function is depicted below.

 \noindent \(\langle\) Function to insert an Org Babel code block \(\rangle \equiv\)
\begin{verbatim}
(defun my-org-babel-insert-src-block ()
  (interactive)
  <<Ask for a name and update variable>>
  <<Update other variables dependent on the name>>
  <<Insert the code block>>)
\end{verbatim}

\subsection{Ask for a name and update variable}
\label{sec:orge5cc8d7}

 \noindent \(\langle\) Block name \(\rangle \equiv\)
\begin{verbatim}
my-org-babel-insert-src-block-name
\end{verbatim}

 \noindent \(\langle\) Ask for a name and update variable \(\rangle \equiv\)
\begin{verbatim}
(setq <<Block name>>
      (read-string "Name: "))
\end{verbatim}

\subsection{Update other variables dependent on the name}
\label{sec:orgb7c2928}

 \noindent \(\langle\) Update other variables dependent on the name \(\rangle \equiv\)
\begin{verbatim}
<<Update section name header variable>>
<<Update noweb-ref block header variable>>
\end{verbatim}

\subsection{Update section name header variable}
\label{sec:org4e2e856}

 \noindent \(\langle\) Section name header \(\rangle \equiv\)
\begin{verbatim}
my-org-babel-insert-src-block-section-name-header
\end{verbatim}

 \noindent \(\langle\) Update section name header variable \(\rangle \equiv\)
\begin{verbatim}
(setq <<Section name header>>
      <<Get the section name header>>)
\end{verbatim}

\subsection{Get the section name header}
\label{sec:orgf2e3702}

 \noindent \(\langle\) Get the section name header \(\rangle \equiv\)
\begin{verbatim}
(if (string= <<Block name>> "") ""
  (concat "@@latex: \\noindent@@ \\langle " <<Block name>> " \\rangle\\equiv"))
\end{verbatim}

\subsection{Update noweb-ref block header variable}
\label{sec:orgaf82aea}

 \noindent \(\langle\) noweb-ref block header \(\rangle \equiv\)
\begin{verbatim}
my-org-babel-insert-src-block-noweb-ref-block-header
\end{verbatim}

 \noindent \(\langle\) Update noweb-ref block header variable \(\rangle \equiv\)
\begin{verbatim}
(setq <<noweb-ref block header>>
      <<Get noweb-ref block header>>)
\end{verbatim}

\subsection{Get noweb-ref block header}
\label{sec:org268726a}

 \noindent \(\langle\) Get noweb-ref block header \(\rangle \equiv\)
\begin{verbatim}
(if (string= <<Block name>> "") ""
  (concat " :noweb-ref " <<Block name>>))
\end{verbatim}

\subsection{Insert the code block}
\label{sec:org198c187}

 \noindent \(\langle\) Insert the code block \(\rangle \equiv\)
\begin{verbatim}
<<Insert the block name, if non-empty>>
<<Insert the section name header, if non-empty>>
(insert "#+begin_src ")
<<Insert the language>>
<<Insert the noweb-ref block header, if non-empty>>
(insert "\n\n#+end_src\n")
(previous-line)(previous-line)
\end{verbatim}

\subsection{Insert the language}
\label{sec:orgec67c22}

Insert the language
 \noindent \(\langle\) Insert the language \(\rangle \equiv\)
\begin{verbatim}
(insert
 (completing-read "Language: "
                  <<Languages>>))
\end{verbatim}

\subsection{Languages}
\label{sec:org195a119}

These are the \href{https://orgmode.org/worg/org-contrib/babel/languages/index.html}{languages supported by Org Babel}, as of November, 2022.

 \noindent \(\langle\) Languages \(\rangle \equiv\)
\begin{verbatim}
(list
 "C"
 "D"
 "F90"
 "R"
 "awk"
 "calc"
 "clojure"
 "comint"
 "cpp"
 "css"
 "ditaa"
 "dot"
 "elisp"
 "emacs-lisp"
 "eshell"
 "forth"
 "gnuplot"
 "groovy"
 "haskell"
 "java"
 "js"
 "julia"
 "latex"
 "lisp"
 "lua"
 "ly"
 "makefile"
 "matlab"
 "max"
 "ocaml"
 "octave"
 "org"
 "perl"
 "plantuml"
 "processing"
 "python"
 "ruby"
 "sass"
 "scheme"
 "screen"
 "sed"
 "shell"
 "sql"
 "sqlite")
\end{verbatim}

\subsection{Insert the block name, if non-empty}
\label{sec:org672cbc3}

 \noindent \(\langle\) Insert the block name, if non-empty \(\rangle \equiv\)
\begin{verbatim}
(unless (string= <<Block name>> "")
  (org-insert-heading)
  (insert <<Block name>> "\n\n"))
\end{verbatim}

\subsection{Insert the section name header, if non-empty}
\label{sec:org6401521}

 \noindent \(\langle\) Insert the section name header, if non-empty \(\rangle \equiv\)
\begin{verbatim}
(unless (string= <<Section name header>> "")
  (insert <<Section name header>> "\n"))
\end{verbatim}

\subsection{Insert the noweb-ref block header, if non-empty}
\label{sec:orgc51bac3}

 \noindent \(\langle\) Insert the noweb-ref block header, if non-empty \(\rangle \equiv\)
\begin{verbatim}
(unless (string= <<noweb-ref block header>> "")
  (insert <<noweb-ref block header>>))
\end{verbatim}

\subsection{Add a keybinding}
\label{sec:orgd793293}

Add a keybinding

 \noindent \(\langle\) Add a keybinding \(\rangle \equiv\)
\begin{verbatim}
(define-key org-mode-map
  (kbd "C-. s")
  'my-org-babel-insert-src-block)
\end{verbatim}

\section{License}
\label{sec:org0d4e676}

This work is dedicated to the public domain.  To the extent possible under law, all copyright and related or neighboring rights to this work are waived worldwide.
\end{document}